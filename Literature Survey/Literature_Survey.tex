\documentclass[12pt,a4paper]{article}
\usepackage{times}
\usepackage{durhampaper}
\usepackage{harvard}
\usepackage{cite}
\usepackage[utf8x]{inputenc}
\usepackage{amsmath}



\citationmode{abbr}
\bibliographystyle{harvard}

\title{Literature Survey}

\author{Molly Hayward}
\student{Molly Hayward}
\supervisor{Dr Noura Al-Moubayed}

\date{23/10/2019}

\begin{document}

\maketitle
\section{Introduction}
\begin{center}
	\textbf{Project title:} Applying machine learning and deep learning techniques to sarcasm detection 
\end{center}



\subsection{Problem Background}
\noindent A form of irony, sarcasm is defined as the use of remarks that clearly mean the opposite of what they say, made in order to hurt someone's feelings or to criticize something in a humorous way []. This presents a fatal problem for classic sentiment analyzers, as text that appears to be positive at surface-level can instead convey an alternate negative meaning. Furthermore, failure to recognize sarcasm can be a human issue too, owing to the fact that its presence (or lack thereof) is considered rather subjective. Abercrombie et al. (2016) \cite{abercrombie2016putting} showed that different humans do not classify statements as either sarcastic or non-sarcastic in the same way - truly, the existence of sarcasm can only be conclusively affirmed by the author. Additionally, developmental differences such as autism, as well as cultural and societal nuances cause variations in the way different people define and perceive sarcasm. \\

\noindent There exists an extensive body of literature covering machine learning and deep learning approaches to natural language processing problems, however in the application domain of sarcasm detection, they mostly display low accuracy. This could be due, in part, to the absence of a concrete, all-encompassing definition of sarcasm. N.B. this may also contribute to poor human accuracy in detecting a fundamentally human construct. \\

\noindent Despite the challenges in this domain, the ultimate aim of this project is to produce a tool that can detect sarcasm with a \textit{high degree} of accuracy. In my endeavour to realise this aim, I will apply state-of-the-art word embedding and classification techniques, using machine learning and deep learning.


\subsection{Terms}


\newpage

\section{Themes}

\subsection{Vectorization of textual data}
\noindent Extracting meaningful data from large corpora is undoubtedly a complex task, however in order to do this successfully, we must first construct word vectors that capture semantic and syntactic relationships in a process known as vectorization. In this section, we examine this process with the aim of answering the question: how can we \textit{best} vectorize text so as to encode the important features of language?

\paragraph{Traditional Approaches --}
\noindent Perhaps the simplest vectorization approach is \textit{Bag of Words (BOW)}, which encodes text as a single vector containing a count of the number of occurrences of each word in the snippet. \textit{Term Frequency-Inverse document frequency (TF-IDF)} \cite{robertson1976relevance} extends the BOW model by providing each word with a score that reflects its level of importance based upon its frequency within the document. Predictably, vectors produced in this way do not preserve the context within which words are found, complicating the task of discerning their meaning. Hence, these approaches are unlikely to be suitable in the application domain of detecting sarcasm - a highly contextual phenomenon. The \textit{Bag of N-grams} approach is a generalisation of Bag of Words, whereby instead of counting the number of occurrences of each unigram, we count the number of occurrences of each N-gram. This allows us to capture a small window of context around each word, however this results in a much larger and sparser feature set.

\paragraph{Word Embedding --}
\noindent First introduced in Mikolov et al. \cite{mikolov2013efficient}, \textit{Word2Vec} describes a group of related models that can be used to produce high-quality vector representations of words - two such models are \textit{Continuous Bag-of-Words} and \textit{Continuous Skip-Gram}. It consists of a shallow (two-layer) neural network that takes a large corpus and produces a high-dimensional vector space. Unlike previous techniques, words that share a similar context tend to have collinear vectors, that is to say they are clustered together in the feature space. Consequently, Word2Vec is able to preserve the semantic relationships between words, even constructing analogies by composing vectors e.g.\ king - man + woman ≈ queen. Likewise, it captures syntactic regularities such as the singular to plural relationship e.g.\ cars - car ≈ apples - apple. In Word2Vec, intrinsic statistical properties of the corpus, which were key to earlier techniques, are neglected, therefore global patterns may be overlooked.To mitigate against this, the \textit{GloVe} \cite{pennington2014glove} approach generates word embeddings by constructing an explicit word co-occurrence matrix for the entire corpus, such that the dot product of two word embeddings is equal to log of the number of times these words co-occur (within a defined window). Despite the ability to preserve semantic relationships between words, Word2Vec and GloVe do not accommodate polysemy, which describes the co-existence of alternative meanings for the same word. In addition, they cannot generalise to words that were not specifically included in the training set. Hence, we may need to use a more robust approach.

\paragraph{Contextualized Word Embedding --}

\newpage

\subsection{Feature extraction from word embeddings}
have seen success in other NLP classification tasks, such as ... In the domain of sarcasm detection, 

\newpage

\subsection{Model Performance Evaluation}
Lastly, evaluating the successes and failures of the trained model is a critical step. A simple approach is to use accuracy which refers to the proportion of data that is correctly labelled as either sarcastic or non-sarcastic. However, sarcasm is a minority class therefore on an unbalanced dataset we could achieve high accuracy by simply labelling every statement as non-sarcastic. In an attempt to mitigate against this, I will instead form a conclusion based on the $F_{1}$ score i.e.\ the harmonic mean of precision and recall, where scores range from 0 (worst) to 1 (best). This metric has faced some criticism for giving equal weight to both precision and recall \cite{hand2018note}, therefore I will consider both measures separately, as well as in combination.

\begin{align*}
\mbox{precision} &= \frac{\mbox{true positives}}{\mbox{true positives + false positives}}   &  \mbox{recall} &= \frac{\mbox{true positives}}{\mbox{true positives + false negatives}}  \\
\end{align*}

In the domain of sarcasm detection, precision refers to the proportion of the classified-sarcastic data that is \textit{truly sarcastic} i.e.\ how many of the positives are true positives, and recall describes the proportion of the truly sarcastic data in the corpus that is \textit{classified} as such i.e.\ how many true positives are labelled as positives. \\


\hrulefill

\bibliography{Literature_Survey}


\end{document}