\documentclass[12pt,a4paper]{article}
\usepackage{times}
\usepackage{durhampaper}
\usepackage{url}
\usepackage{harvard}
\usepackage{cite}
\citationmode{abbr}
\bibliographystyle{plain}

\title{An Analysis of Machine Learning and Deep Learning Techniques for the Detection of Sarcasm in Online Product Reviews}
\author{} % leave; your name goes into \student{}
\student{Molly Hayward}
\supervisor{Dr Noura Al-Moubayed}
\degree{BSc Computer Science}

\date{}

\begin{document}

\maketitle

\begin{abstract}
\\ \indent \textbf{Context / Background --} 
Sarcasm is a powerful linguistic anomaly that when present in text, can alter its meaning entirely. Detecting sarcasm proves a significant challenge for traditional sentiment analysers. This highlights the scope for innovative machine learning and deep learning solutions to this complex problem.

\indent \textbf{Aims --} Despite the challenges in this domain, the ultimate aim of this project is to produce a tool that can detect sarcasm with a \textit{high degree} of accuracy.

\indent \textbf{Method --} In my endeavour to realise this aim, I implemented state-of-the-art word embedding and text classification techniques.

\indent \textbf{Results --} Through extensive experimentation, I found that

\indent \textbf{Conclusions --} Following this experimentation, I conclude that

This section should not be longer than half of a page, and having no more than one or two sentences under each heading is advised. Do not cite references in the abstract.
\end{abstract}

\begin{keywords}
Machine learning, Deep learning, Sarcasm detection
\end{keywords}

\newpage

\section{Introduction}
\noindent Sarcasm is defined as the use of remarks that clearly mean the opposite of what they say, made in order to hurt someone's feelings or to criticize something in a humorous way \cite{cambridgesarcasm2020}. Sarcasm poses a significant challenge within the field of natural language processing, specifically within sentiment analysis, as it transforms the sentiment polarity of a positive or negative utterance into its opposite. Sentiment analysis is the task of deducing the sentiment polarity of text - typically whether the author is in favour of, or against, a certain subject. It is increasingly common for organisations to use sentiment analysis in order to gauge public opinion on their products and services; however, classic sentiment analysers cannot deduce the implicit meaning of sarcastic text and will wrongly classify the author's opinion. This phenonmenon of sentiment incongruity often lies at the heart of sarcasm, therefore advancements in automatic sarcasm detection research have the potential to vastly improve the sentiment analysis task.\\


\noindent As sarcasm is multi-faceted, this makes its detection a unique and interesting challenge. It can be both explicit and implicit; oftentimes, contextual cues are more powerful indicators of sarcasm than the words themselves. However we often lose this context, and hence the sarcastic undertones, in transcription. Consider the scenario where a person is congratulated on their hard work, despite the obviousness of them having not worked hard at all. Or perhaps a customer thanking a waiter for the delicious food, even though they sent it back to the kitchen. In isolation, the speech is not enough to convey the sarcastic intent. Furthermore, even humans struggle to consistently recognize sarcastic intent; due in part to the lack of an all-encompassing, universal definition of sarcasm. In a 2011 study, Gonz{\'a}lez-Ib{\'a}nez et al. \cite{gonzalez2011identifying} found low agreement rates between human annotators when classifying statements as either sarcastic or non-sarcastic, and in their second study, three annotators unanimously agreed on a label less than 72\% of the time. Truly, the existence of sarcasm can only be conclusively affirmed by the author. Additionally, developmental differences such as autism, as well as cultural and societal nuances cause variations in the way different people define and perceive sarcasm.\\


All of these components make sarcasm detection an extremely complex task for both humans and computers. Despite this, most humans can recognise sarcasm most of the time. If we could replicate this performance, or even perhaps improve upon it, we could move towards a more concrete definition of what \textit{makes} a statement sarcastic. This highlights the scope for automating this process, and the need for an innovative solution.\\


\noindent The \textbf{research questions} guiding this project are as follows --
\\ \indent \textit{Which linguistic cues indicate sarcastic intent in written text? How can a model be used to detect the words that correlate more to sarcastic labels? Do deep learning techniques perform better than machine learning approaches for sarcasm detection? Can the detection of sarcasm improve the sentiment analysis task? Does the proposed solution perform well on other datasets?}\\

This section briefly introduces the general project background, the research question you are addressing, and the project objectives.  It should be between 2 to 3 pages in length.\\



\section{Related Work}
In similar studies, the terms irony and sarcasm are often used interchangeably \cite{tsur2010icwsm}. Hence, I have provided an overview of their definitions. Online, it is commonly found in opinion-based user generated content.

\begin{center}
	Table 1: Definition of subjective terms
\end{center}
\begin{tabular}{p{3cm}p{8cm}p{5cm}}
	\hline
	\textbf{Term} & \textbf{Definition} & \textbf{Example}\\
	\hline\hline
	Irony & The use of words that are the opposite of what you mean, as a way of being funny. \cite{cambridgeirony2020} &  Example of irony\\
	\hline
	Sarcasm & The use of remarks that clearly mean the opposite of what they say, made in order to hurt someone's feelings or to criticize something in a humorous way. \cite{cambridgesarcasm2020} &  Example of sarcasm\\
	\hline
\end{tabular}\\

\noindent Most approaches treat sarcasm detection as a binary classification problem, in which text is grouped into two categories - sarcastic and non-sarcastic. In the following text, I will compare methods used specifically in sarcasm detection research, as well as techniques that have seen success in other NLP classification tasks as they may have potential to perform well in this specific domain.

\paragraph{Traditional Classifiers --}
A number of simple linguistic approaches have been used in previous research into sarcasm detection. One such class of naive approaches is \textit{rule-based}, where text is classified based on a set of linguistic rules. Maynard and Greenwood (2014) \cite{maynard2014cares} proposed that the sentiment contained in hashtags can indicate the prescence of sarcasm in tweets, such as \#notreally in the example "I love doing the washing-up \#notreally". They used a rule-based approach to determine that if the sentiment of a tweet contradicts that of the hashtag, then this tweet is an example of sarcasm. In Riloff et al. (2013) \cite{riloff2013sarcasm}, sarcasm is described as a contrast between positive sentiment and negative situation. This description is leveraged by Bharti et al (2015) \cite{bharti2015parsing}, which presents two rule-based approaches to sarcasm detection. The first approach identifies sentiment bearing situation phrases, classifying the phrase as sarcastic if the negative situation is juxtaposed with a positive sentence. Rule-based techniques are fairly primitive when compared to their modern, deep learning counterparts. Especially when you consider that each ruleset must be generated manually and for each dataset and domain.


\paragraph{Machine-Learning Classifiers --}
Using labelled training data, a machine-learning algorithm can learn patterns that are associated with each category of data. This allows it to classify unseen text into these categories without the need for a static linguistic rule set. If the training data is high-quality and plentiful, the model can begin to make accurate predictions. There are a few general categories of machine-learning classifiers, including Naive Bayes, Support Vector machines (SVMs) and logistic regression based classifiers. Although, not all approaches fit neatly into these categories.

For example, Tsur et al. (2010) \cite{tsur2010icwsm} used a semi-supervised algorithm, \textit{SASI}, in order to detect sarcasm in online product reviews. Their dataset contained 66000 Amazon reviews, each annotated by three humans as a result of crowdsourcing. They extracted syntactic and pattern-based features from the corpus, then used a classifcation strategy similar to k-nearest neighbor (kNN), whereby they assigned a score to each feature vector in the test set by computing the weighted average of the k-closest neighbors in the training set N.B. neighbours are vectors that share at least one pattern feature. This model had relative success - achieving a precision of 77\% and recall of 83.1\% on newly discovered sentences.

Naive Bayes classifiers are supervised algorithms that use bayes theorem to make predictions. Reyes et al. (2013) \cite{reyes2013multidimensional} used naive bayes and decision trees algorithm in order to detect irony which is closely related to sarcasm. They experimented with balanced and unbalanced distributions (25\% ironic tweets and 75\% other), achieving an F-score of 0.72 on the balanced distribution, dropping to 0.53 for the inbalanced distribution. In a similar vein, Barbieri et al. (2014) \cite{barbieri2014modelling} used random forest and decision tree classifiers, also for the detection of irony. They used six types of features in order to represent tweets, and recorded results over three categories of training data - education, humor and politics.

Support Vector Machines (SVM), can be effective when smaller amounts of training data are available, however they require more computational resources than naive bayes. SVMs use kernel functions to draw hyperplanes dividing a space into subspaces. Gonz{\'a}lez-Ib{\'a}nez et al\. (2011) \cite{gonzalez2011identifying} used two classifiers, support vector machine with sequential minimal optimization, as well as logistic regression. Their best result was an accuracy of 0.65, achieved with the combination of support vector machine with sequential miniminal optimization and unigrams. \textit{Pt{\'a}{\v{c}ek et al. (2014)}} \cite{ptavcek2014sarcasm} also used support vector machine and Maximum Entropy classifiers. They also performed classification on balanced and inbalanced distributions.

\paragraph{Deep-Learning Classifiers --}
Deep neural networks (DNNs) are increasingly being used in text classification tasks \cite{zhang2015character, poria2016deeper}. Two common DNN architectures are Recurrent Neural Networks (RNNs) and Convolutional Neural Networks (CNNs). They typically require a lot more training data than machine learning algorithms, but have been shown to produce state-of-the-art results in several domains.

Recurrent Neural Networks are good at processing sequential, or time-series, information such as text. This is due to the fact that RNNs have a 'memory', in that the input to the current step is the output from the previous step. However, they suffer from the vanishing gradient problem. This is where gradient values get smaller as it backtracks through lower layers, making it harder for the network to update weights and causing calculations to take longer. Recurrent Neural Networks are not very good at remembering long term dependencies, therefore in this instance we can use \textit{Long short-term memory (LSTM)} models \cite{hochreiter1997long} instead which are more effective at preserving long term dependencies.

One particularly interesting technique is the \textit{Hierarchical Attention Network (HAN)} defined in Yang et al. (2016) \cite{yang2016hierarchical} - an attention-based LSTM. The intuition is that certain words and sentences in a document contribute more to its overall meaning, and this is highly context dependent. They included one attention mechanism at word-level and another at sentence-level, and this allowed them to determine which words and phrases correlate more to certain labels. Using a similar approach in the domain of sarcasm may allow me to highlight the attention-words that are strongly influence the level of sarcasm. A similar technique was applied in \textit{Ghosh et al. 2018)} \cite{ghosh2018sarcasm} where they concluded that emoticons such as ':)' and interjections e.g. "ah", "hmm" correspond with highly weighted sentences. 
Gated Recurrent Units (GRUs) are another type of RNN used to overcome the vanishing gradient problem.  \textit{Zhang et al. (2016)} \cite{zhang2016tweet} used a bidirectional gated RNN to capture local information in tweets, as well as a pooling neural network to extract context from historic tweets, achieving 78.55\% accuracy.

Convolutional Neural Networks allow us to extract higher-level features, and they are based on the mathematical operation of convolution. \textit{Zhang et al. (2015)}  \cite{zhang2015character} explored the used of character-level convolutional neural networks for text classification. This network consists of 6 convolutional layers and 3 fully-connected layers. They found that it showed better peformance on raw texts such as an Amazon product review corpus. CNNs have been previously used in sarcasm detection. For example, \textit{Poria et al. (2017)} \cite{poria2016deeper} describes a convolutional neural network (CNN) for sarcasm detection. \\

\vfill


%This section presents a survey of existing work on the problems that this project addresses.  it should be between 2 to 4 pages in length.  The rest of this section shows the formats of subsections as well as some general formatting information for tables, figures, references and equations. Note that the whole report, including the references, should not be longer than 20 pages in length.  The system will not accept any report longer than 20 pages.  It should be noted that not all the details of the work carried out in the project can be represented in 20 pages.  It is therefore vital that the Project Log book be kept up to date as this will be used as supplementary material when the project paper is marked.  There should be between 10 and 20 referenced papers---references to Web based pages should be less than 10\%.




\begin{tabular}{p{4cm}p{3cm}p{2cm}}
	\hline
	\textbf{Title} & \textbf{Size} & \textbf{Mean Length}\\
	\hline\hline
	News Headlines Dataset For Sarcasm Detection & \begin{tabular}{@{}c@{}@{}} \\26709 instances \\ +ve: 43.9\% \\ -ve: 56.1\% \end{tabular} &  9.85 words\\
	\hline
	SARC & \begin{tabular}{@{}c@{}@{}} \\1010826 instances \\ +ve: 50\% \\ -ve: 50\% \end{tabular} &  10.46 words\\
	\hline
	Sarcasm Amazon Reviews Corpus & \begin{tabular}{@{}c@{}@{}} \\1010826 instances \\ +ve: 50\% \\ -ve: 50\% \end{tabular} &  10.46 words\\
	\hline
\end{tabular}\\


%The font used for the main text should be Times New Roman (Times) and the font size should be 12.  The first line of all paragraphs should be indented by 0.25in, except for the first paragraph of each section, subsection, subsubsection etc. (the paragraph immediately after the header) where no indentation is needed.


%In general, figures and tables should not appear before they are cited.  Place figure captions below the figures; place table titles above the tables.  If your figure has two parts, for example, include the labels ``(a)'' and ``(b)'' as part of the artwork.  Please verify that figures and tables you mention in the text actually exist.  make sure that all tables and figures are numbered as shown in Table \ref{units} and Figure 1.
%sort out your own preferred means of inserting figures


%The list of cited references should appear at the end of the report, ordered alphabetically by the surnames of the first authors.  References cited in the main text should use Harvard (author, date) format.  When citing a section in a book, please give the relevant page numbers, as in \cite[p293]{budgen}.  When citing, where there are either one or two authors, use the names, but if there are more than two, give the first one and use ``et al.'' as in  , except where this would be ambiguous, in which case use all author names.


%You need to give all authors' names in each reference.  Do not use ``et al.'' unless there are more than five authors.  Papers that have not been published should be cited as ``unpublished'' \cite{euther}.  Papers that have been submitted or accepted for publication should be cited as ``submitted for publication'' as in \cite{futher} .  You can also cite using just the year when the author's name appears in the text, as in ``but according to Futher \citeyear{futher}, we \dots''.  Where an authors has more than one publication in a year, add `a', `b' etc. after the year.

\newpage
\section{Solution}


Observing that people are inclined to be more sarcastic towards specific subjects such as weather or work.
Feed features to trainer such as support vector machine.
Represent input words numerically.
One-hot encoding has two main disadvantages, firstly, each vector is the size of the vocabulary, with a single 1 in the column marking the word. This creates sparse inefficient representations.


\noindent There exists an extensive body of literature covering machine learning and deep learning approaches to natural language processing problems, however in the application domain of sarcasm detection, they mostly display low accuracy.\\

\subsection{Data pre-processing}
\noindent Data pre-processing is an important stage in most NLP tasks, with the potential to hinder or boost model performance. Datasets in this domain are especially messy, often because they are collated from user generated content. Hence, data pre-processing is vital to reduce noise and sparsity in the feature space caused by inconsistent letter capitalization, varying use of punctuation and erroneous spelling. The suitability of each technique is highly dependent on the domain, as for some datasets, certain types of pre-processing may result in the removal of useful features.





On twitter, users can include hyperlinks to other websites in their tweets, and this can be a source of noise in the dataset. \textit{Pt{\'a}{\v{c}ek et al. (2014)}} \cite{ptavcek2014sarcasm} collated a dataset of sarcastic and non-sarcastic tweets, whereby the presence of sarcasm is indicated by the marker \#sarcasm, removing URLS and references to users, as well as hashtags. However, \textit{Liebrecht et al. (2013)} \cite{liebrecht2013perfect} showed that data contained in hashtags can be used to indicate sarcasm, such as \#not. Transforming text into lowercase is a common pre-processing strategy, useful for reducing sparsity caused by variations in letter capatiliztion e.g. 'Cat', 'cAt', and 'CAt' are all mapped to the same word - 'cat'. Similarly, removing punctuation and extending contractions (e.g. don't $\rightarrow$ do not) is commonplace. However, punctuation and capitalization can be used for emphasis, therefore removing these attributes may make the presence of sarcasm less obvious.

scraped from Twitter or Amazon product reviews, where spelling errors are rife
\vfill

%This section presents the solutions to the problems in detail.  The design and implementation details should all be placed in this section.  You may create a number of subsections, each focussing on one issue.  

%This section should be between 4 to 7 pages in length.



\section{Results}

%this section presents the results of the solutions.  It should include information on experimental settings.  The results should demonstrate the claimed benefits/disadvantages of the proposed solutions.

%This section should be between 2 to 3 pages in length.

\section{Evaluation}

%This section should between 1 to 2 pages in length.

\section{Conclusions}

%This section summarises the main points of this paper.  Do not replicate the abstract as the conclusion.  A conclusion might elaborate on the importance of the work or suggest applications and extensions.  This section should be no more than 1 page in length.

%The page lengths given for each section are indicative and will vary from project to project but should not exceed the upper limit.  A summary is shown in Table \ref{summary}.

\begin{table}[htb]
\centering
\caption{SUMMARY OF PAGE LENGTHS FOR SECTIONS}
\vspace*{6pt}
\label{summary}
\begin{tabular}{|ll|c|} \hline
& \multicolumn{1}{c|}{\bf Section} & {\bf Number of Pages} \\ \hline
I. & Introduction & 2--3 \\ \hline
II. & Related Work & 2--3 \\ \hline
III. & Solution & 4--7 \\ \hline
IV. & Results & 2--3 \\ \hline
V. & Evaluation & 1-2 \\ \hline
VI. & Conclusions & 1 \\ \hline
\end{tabular}
\end{table}


\bibliography{FinalPaper}


\end{document}