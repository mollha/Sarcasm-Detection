\documentclass[12pt,a4paper]{article}
\usepackage{times}
\usepackage{durhampaper}
\usepackage{url}
\usepackage{harvard}
\usepackage{cite}
\citationmode{abbr}
\bibliographystyle{plain}

\title{An Analysis of Machine Learning and Deep Learning Techniques for the Detection of Sarcasm in Text}
\author{} % leave; your name goes into \student{}
\student{Molly Hayward}
\supervisor{Dr Noura Al-Moubayed}
\degree{BSc Computer Science}

\date{}

\begin{document}

\maketitle

\begin{abstract}
\\ \indent \textbf{Context / Background --} 
Sarcasm is a powerful linguistic anomaly that when present in text, conveys an opposing message to its literal meaning. It's detection proves a significant challenge for traditional sentiment analysers, as well as in the wider field of natural language processing. This highlights the scope for innovative AI-based solutions to this complex problem.

\indent \textbf{Aims --} Despite the challenges in this domain, the ultimate aim of this project is to produce a tool that can detect sarcasm with a \textit{high degree} of accuracy.

\indent \textbf{Method --} In my endeavour to realise this aim, I implemented state-of-the-art word embedding and text classification techniques.

\indent \textbf{Results --} Through extensive experimentation, I found that

\indent \textbf{Conclusions --} Following this experimentation, I conclude that

This section should not be longer than half of a page, and having no more than one or two sentences under each heading is advised. Do not cite references in the abstract.
\end{abstract}

\begin{keywords}
Machine learning, Deep learning, Sarcasm detection
\end{keywords}

\section{Introduction}
This section briefly introduces the general project background, the research question you are addressing, and the project objectives.  It should be between 2 to 3 pages in length.\\

Detecting sarcasm has proven insurmountably difficult. Due in part, to a lack of a concrete, all-encompassing definition.

Sarcasm has no set rules describing its formulation. So prevalent are these  discrepancies, even humans cannot recognise sarcasm in all of its forms.


As more data becomes available, the need to automatically 

Misinterpreting sarcasm as a literal statement 

Humans struggle with this task. 

This highlights the scope for automating this process.


\noindent \textbf{Research questions} guiding this project are as follows --
\\ \indent \textit{Do deep learning techniques perform better than machine learning approaches for sarcasm detection? Can the detection of sarcasm improve the sentiment analysis task? How can a model be used to detect the words that correlate more to sarcastic labels? Does the proposed solution perform well on other datasets?}\\


In similar studies, the terms irony and sarcasm are used interchangeably. Hence, I have provided an overview of their definitions, accompanied by real examples found among the datasets used in this project.

\begin{center}
	Table 1: Definition of subjective terms
\end{center}
\begin{tabular}{p{3cm}p{8cm}p{5cm}}
	\hline
	\textbf{Term} & \textbf{Definition} & \textbf{Example}\\
	\hline\hline
	Irony & The use of words that are the opposite of what you mean, as a way of being funny. \cite{cambridgeirony2019} &  Example of irony\\
	\hline
	Sarcasm & The use of remarks that clearly mean the opposite of what they say, made in order to hurt someone's feelings or to criticize something in a humorous way. \cite{cambridgesarcasm2019} &  Example of sarcasm\\
	\hline
\end{tabular}\\


Note that the whole report, including the references, should not be longer than 20 pages in length.  The system will not accept any report longer than 20 pages.  It should be noted that not all the details of the work carried out in the project can be represented in 20 pages.  It is therefore vital that the Project Log book be kept up to date as this will be used as supplementary material when the project paper is marked.  There should be between 10 and 20 referenced papers---references to Web based pages should be less than 10\%.

\section{Related Work}
\noindent There exists an extensive body of literature covering machine learning and deep learning approaches to natural language processing problems, however in the application domain of sarcasm detection, they mostly display low accuracy. This could be due, in part, to the absence of a concrete, all-encompassing definition of sarcasm. N.B. this may also contribute to poor human accuracy in detecting a fundamentally human construct. \\


This section presents a survey of existing work on the problems that this project addresses.  it should be between 2 to 4 pages in length.  The rest of this section shows the formats of subsections as well as some general formatting information for tables, figures, references and equations.

\subsection{Main Text}
\begin{tabular}{p{4cm}p{3cm}p{2cm}}
	\hline
	\textbf{Title} & \textbf{Size} & \textbf{Mean Length}\\
	\hline\hline
	News Headlines Dataset For Sarcasm Detection & \begin{tabular}{@{}c@{}@{}} \\26709 instances \\ +ve: 43.9\% \\ -ve: 56.1\% \end{tabular} &  9.85 words\\
	\hline
	SARC & \begin{tabular}{@{}c@{}@{}} \\1010826 instances \\ +ve: 50\% \\ -ve: 50\% \end{tabular} &  10.46 words\\
	\hline
	Sarcasm Amazon Reviews Corpus & \begin{tabular}{@{}c@{}@{}} \\1010826 instances \\ +ve: 50\% \\ -ve: 50\% \end{tabular} &  10.46 words\\
	\hline
\end{tabular}\\


The font used for the main text should be Times New Roman (Times) and the font size should be 12.  The first line of all paragraphs should be indented by 0.25in, except for the first paragraph of each section, subsection, subsubsection etc. (the paragraph immediately after the header) where no indentation is needed.

\subsection{Figures and Tables}
In general, figures and tables should not appear before they are cited.  Place figure captions below the figures; place table titles above the tables.  If your figure has two parts, for example, include the labels ``(a)'' and ``(b)'' as part of the artwork.  Please verify that figures and tables you mention in the text actually exist.  make sure that all tables and figures are numbered as shown in Table \ref{units} and Figure 1.
%sort out your own preferred means of inserting figures

\begin{table}[htb]
\centering
\caption{UNITS FOR MAGNETIC PROPERTIES}
\vspace*{6pt}
\label{units}
\begin{tabular}{ccc}\hline\hline
Symbol & Quantity & Conversion from Gaussian \\ \hline
\end{tabular}
\end{table}

\subsection{References}

The list of cited references should appear at the end of the report, ordered alphabetically by the surnames of the first authors.  References cited in the main text should use Harvard (author, date) format.  When citing a section in a book, please give the relevant page numbers, as in \cite[p293]{budgen}.  When citing, where there are either one or two authors, use the names, but if there are more than two, give the first one and use ``et al.'' as in  , except where this would be ambiguous, in which case use all author names.

You need to give all authors' names in each reference.  Do not use ``et al.'' unless there are more than five authors.  Papers that have not been published should be cited as ``unpublished'' \cite{euther}.  Papers that have been submitted or accepted for publication should be cited as ``submitted for publication'' as in \cite{futher} .  You can also cite using just the year when the author's name appears in the text, as in ``but according to Futher \citeyear{futher}, we \dots''.  Where an authors has more than one publication in a year, add `a', `b' etc. after the year.

\section{Solution}

This section presents the solutions to the problems in detail.  The design and implementation details should all be placed in this section.  You may create a number of subsections, each focussing on one issue.  

This section should be between 4 to 7 pages in length.

\section{Results}

this section presents the results of the solutions.  It should include information on experimental settings.  The results should demonstrate the claimed benefits/disadvantages of the proposed solutions.

This section should be between 2 to 3 pages in length.

\section{Evaluation}

This section should between 1 to 2 pages in length.

\section{Conclusions}

This section summarises the main points of this paper.  Do not replicate the abstract as the conclusion.  A conclusion might elaborate on the importance of the work or suggest applications and extensions.  This section should be no more than 1 page in length.

The page lengths given for each section are indicative and will vary from project to project but should not exceed the upper limit.  A summary is shown in Table \ref{summary}.

\begin{table}[htb]
\centering
\caption{SUMMARY OF PAGE LENGTHS FOR SECTIONS}
\vspace*{6pt}
\label{summary}
\begin{tabular}{|ll|c|} \hline
& \multicolumn{1}{c|}{\bf Section} & {\bf Number of Pages} \\ \hline
I. & Introduction & 2--3 \\ \hline
II. & Related Work & 2--3 \\ \hline
III. & Solution & 4--7 \\ \hline
IV. & Results & 2--3 \\ \hline
V. & Evaluation & 1-2 \\ \hline
VI. & Conclusions & 1 \\ \hline
\end{tabular}
\end{table}


\bibliography{projectpaper}


\end{document}